\documentclass[fleqn,10pt]{wlscirep}
\title{Assessing heterogeneity (and predictability ??) of runners' performance in Switzerland}

\author[1,*]{Gianrocco Lazzari}
\author[2]{Stefano Savaré}
\author[2]{Antonio Iubatti}
\author[2]{Maxime Peschard}
\author[2]{Ondine Chanon}
\author[3]{Michele Catasta}
\author[1]{Marcel Salathé}

\affil[1]{Global Health Institute, School of Life Sciences, Ecole Polytechnique Fédérale de Lausanne (EPFL), Lausanne, Switzerland.}
\affil[2]{School of Computer Science, Ecole Polytechnique Fédérale de Lausanne (EPFL), Lausanne, Switzerland.}

\affil[3]{Department of Computer Science, Stanford University, Stanford, USA.}


\affil[*]{gianrocco.lazzari@epfl.ch}

%\affil[+]{these authors contributed equally to this work}

%\keywords{Aging, Distance running, Endurance performance, Sex difference}

%%%% %%%% %%%% %%%%  GR's packages %%%% %%%% %%%% %%%% %%%% 

\usepackage{caption,subfigure,float}
%\usepackage{subcaption} 

% for an examples please fer to http://tex.stackexchange.com/questions/148438/putting-two-images-beside-each-other

%packages for references
\usepackage{hyperref,url,cite}


%%%% %%%% %%%% %%%% %%%% %%%% %%%% %%%% %%%% %%%% %%%% %%%% 

\begin{abstract}
	
%Example Abstract. Abstract must be under 200 words and not include subheadings or citations. 



	\textbf{keywords}: Aging, Distance running, Endurance performance, Sex difference

\end{abstract}
\begin{document}

\flushbottom
\maketitle
% * <john.hammersley@gmail.com> 2015-02-09T12:07:31.197Z:
%
%  Click the title above to edit the author information and abstract
%
\thispagestyle{empty}

%\noindent Please note: Abbreviations should be introduced at the first mention in the main text – no abbreviations lists. Suggested structure of main text (not enforced) is provided below.

\section*{Introduction}

%The Introduction section, of referenced text\cite{Figueredo:2009dg} expands on the background of the work (some overlap with the Abstract is acceptable). The introduction should not include subheadings.

as show in \cite{connick2015relative} blabalblab 

\section*{Results}

%Up to three levels of \textbf{subheading} are permitted. Subheadings should not be numbered.
%
%\subsection*{Subsection}
%
%Example text under a subsection. Bulleted lists may be used where appropriate, e.g.
%
%\begin{itemize}
%\item First item
%\item Second item
%\end{itemize}
%
%\subsubsection*{Third-level section}
% 
%Topical subheadings are allowed.

	\subsection*{Demographics}
	
		In fig. \ref{participation_overall_by_distance} and \ref{participation_overall_by_gender} we show respectively how the number of runners increased in the last 15 years, by distances and gender. 
		This raise was steeper for man than for women (fig. \ref{participation_overall_by_gender}), and steeper in the shorter distance (10 Km) than in the longer ones ( fig. \ref{participation_overall_by_distance} - participants in full marathons seems to have decreased though) \footnote{For simplicity we only include the most popular distances. 
		There are many events that include shorter distances, like 3 Km, 5 Km, usually attended by a small fraction of young runners.}.
	
		\begin{figure}[h]	
	
			\centering
			
			\subfigure[]{\includegraphics[scale=0.8]{../data_analysis/plots_for_paper/participation_overall_by_distance.pdf}\label{participation_overall_by_distance}}
%					\hfill
			\subfigure[]{\includegraphics[scale=0.8]{../data_analysis/plots_for_paper/participation_overall_by_gender.pdf}\label{participation_overall_by_gender}}
			
			\caption{Number of participants in running competition in Switzerland, across time, by distance (a) and gender (b).}
		
		\end{figure}		
	
		For a significant analysis of runners' performance, it is important to check the amount of data present in the various events, and therefore first assess the heterogeneity of  events popularity.
		In fig. \ref{dist_number_editions_race} we show the  distribution of number of editions each event was hosted, across all history. Counting all editions of  all races as independent, we recorded 222 events. 
		More interestingly in fig. \ref{number_runners_per_race} one can see the broad distribution of number of participants in the different events. In particular, a power-law fit  ( $ f(x) \sim x^{-\alpha} $) provides an exponent of $ \alpha = 1.69 \pm 0.05 $\footnote{The power-law starts from  a lower-bound, whose value results as well from the fitting procedure: $ x_{min} = 688  $ runners/race}.
		
		\begin{figure}[h]	
			\centering
			
			\subfigure[]{\includegraphics[scale=0.4]{../data_analysis/plots_for_paper/dist_number_editions_race.pdf}\label{dist_number_editions_race}}
			%					\hfill
			\subfigure[]{\includegraphics[scale=0.4]{../data_analysis/plots_for_paper/number_runners_per_race.pdf}\label{number_runners_per_race}}
			
			\caption{Assess popularity of running competitions, in Switzerland.}
		\end{figure}		
	
		Inversely, one can measure how participative runners have been across Switzerland. 
		In fig. \ref{number_races_per_runner} we show the distribution\footnote{		One can see that a log-normal would fit better in this case than a power-law model. In particular the fitted parameters are $ \mu = -0.70, \sigma  = 1.55 $} of  the number of events to which each runners participated. We collected data from  a total number of 531426 runners.

		
		
		\begin{figure}[h]	
			\centering
			
			\subfigure[]{\includegraphics[scale=0.4]{../data_analysis/plots_for_paper/number_races_per_runner.pdf}\label{number_races_per_runner}}
			%					\hfill
%			\subfigure[]{\includegraphics[scale=0.4]{../data_analysis/plots_for_paper/}\label{}}
			
			\caption{Assess how participative runners are across competitions, in Switzerland.}
		\end{figure}					
			
	\subsection*{The case of Lausanne Marathon}
		
		We consider the case of 2016 Lausanne Marathon event to present some relevant statistics on age and performance distribution, due to his popularity across all Switzerland (see indeed also fig. \ref{origin_towns_dist} for the distribution of runners' origin towns).
		Apart from the 812 kids (younger than 14 years) running  a shorter race, the 2016 edition had more than 10 thousands participants, divided per race  and gender, as shown in table \ref{counts_per_distance}. 


% thanks to http://truben.no/table/

	 	\begin{table}
	 		\centering
	 		\begin{tabular}[t]{l|l}
	 			Category     & Number of participants \\ \hline
	 			10 km         & 5515                   \\
	 			half-marathon & 4414                   \\
	 			marathon      & 1318                   \\
	 		\end{tabular}
 			\hfill
 			\begin{tabular}[t]{l|l}
				Gender     		& Number of participants \\ \hline
 				M       	  & 6905 					\\
				F		 		& 4655                   \\
 			\end{tabular}
 		\caption{Distribution of number of participants in 2016 Lausanne Marathon, by distance and gender.}
 		\label{counts_per_distance}
	 	\end{table}
 	
 		In fig. \ref{women_age_dist_lausanne} and \ref{men_age_dist_lausanne} we show the distribution of runners' ages in the three main races and divided by gender.
 		For women, even if the median age increases with the distance, begin respectively 
 		34 y.o. (10 Km),
 		36 y.o. (21 Km) and
 		38 y.o. (42 Km),
 		only the women running the 10 Km were 
 		significantly\footnote{All reported tests are done using two-tailed p-value on \href{https://en.wikipedia.org/wiki/Kolmogorov–Smirnov\_test\#Two-sample\_Kolmogorov.E2.80.93Smirnov_test}{Kolmogorov-Smirnov test} $ p \leq 10^{-3} $ .} 
 		younger.
 		For men, median ages also slightly increases with distance
 		39 y.o. (10 Km),
 		39 y.o. (21 Km) and
 		42 y.o. (42 Km),
 		but only the men running the full marathon are significantly older.
 		Furthermore, women are significantly younger than men in all three races.
 		
		\begin{figure}[h]	
			\centering
			
			\subfigure[]{\includegraphics[scale=0.4]{../data_analysis/plots_for_paper/women_age_dist_lausanne.pdf}\label{women_age_dist_lausanne}}
			%					\hfill
			\subfigure[]{\includegraphics[scale=0.4]{../data_analysis/plots_for_paper/men_age_dist_lausanne.pdf}\label{men_age_dist_lausanne}}
			
			\caption{Runners' age distribution in 2016 Lausanne Marathon, for women (a) and men (b) and divided by race (color-coded). For clarity we only plot the Gaussian kernel estimate of the distribution.}
		\end{figure}		 		
		 		
 		
 		In fig. \ref{10km_perf_dist_lausanne}, \ref{half_marathon_perf_dist_lausanne} and \ref{marathon_perf_dist_lausanne} we compare the distribution of performances by gender,  for each race. We found the men were significantly faster in all races.
 				 		
		\begin{figure}[h]	
 			\centering
 			
 			\subfigure[]{\includegraphics[scale=0.4]{../data_analysis/plots_for_paper/10km_perf_dist_lausanne.pdf}\label{10km_perf_dist_lausanne}}
 			%					\hfill
 			\subfigure[]{\includegraphics[scale=0.4]{../data_analysis/plots_for_paper/half_marathon_perf_dist_lausanne.pdf}\label{half_marathon_perf_dist_lausanne}}
 			
 			\subfigure[]{\includegraphics[scale=0.4]{../data_analysis/plots_for_paper/marathon_perf_dist_lausanne.pdf}\label{marathon_perf_dist_lausanne}}
 			
 			\caption{Runners' performance distribution in 2016 Lausanne Marathon, for 10 Km (a), half marathon (b) and marathon (c), divided by gender (color-coded).}
 		\end{figure}		 		
 	
 	
 	
 		It is  also worthy to compare the \textit{pace}-$ \pi $ (min/Km) across different races. 
 		One of course would expect that longer races require higher pace ($ \pi_{10} < \pi_{21} < \pi_{42} $). 
 		These differences were found to be significant for men \ref{men_pace_dist_lausanne} for all races and for women running  \ref{women_pace_dist_lausanne} the marathon (difference $ \langle \pi_{10}  \rangle -  \langle \pi_{21}  \rangle $ for women was not significant)
 		
		\begin{figure}[h]	
 			\centering
 			
 			\subfigure[]{\includegraphics[scale=0.45]{../data_analysis/plots_for_paper/women_pace_dist_lausanne.pdf}\label{women_pace_dist_lausanne}}
 			%					\hfill
 			\subfigure[]{\includegraphics[scale=0.45]{../data_analysis/plots_for_paper/men_pace_dist_lausanne.pdf}\label{men_pace_dist_lausanne}}
 		
 			\caption{Runners' pace distribution in 2016 Lausanne Marathon, for women (a) and men (b), divided by distance (color-coded).}
 		\end{figure}		 		
	
	
	\subsection*{Overall performance analysis}
	
		\subsubsection*{Age-performance relation}
		
		
			We wanted to check whether the previously found	\cite{connick2015relative,knechtle2014relationship,lara2014relationship,lehto2016effects}
			U-relation between age and performance holds as well in the case of swiss races. 
			In fig. \ref{performance_VS_age_marathon} 
			(\textbf{plot to be improved!}) 
			we show the dependence of runners' performance on age, for four of the most popular swiss marathons. 
			The above-mentioned U-shaped although still slightly appearing in the longest distance (42Km), it does emerge more clearly in the half-marathons, as  shown indeed in fig. \ref{performance_VS_age_semi_marathon}
			(\textbf{plot to be improved!}) 
			
			
			\begin{figure}[h]	
		
				\centering
				
				\includegraphics[scale=0.6]{../data_analysis/plots_for_paper/performance_VS_age_marathon.pdf}
				
				%				\subfigure[]{\includegraphics[scale=0.35]{../data_analysis/plots_for_paper/performance_VS_age_marathon.pdf}\label{performance_VS_age_marathon}}
				%					\hfill
				%						\subfigure[]{\includegraphics[scale=0.35]{../data_analysis/plots_for_paper/performance_VS_age_semi_marathon.pdf}\label{performance_VS_age_semi_marathon}}
				
				\caption{Relation between runners' performance (time in minutes to complete the race) and age, for the most popular marathons, color coded by gender.}
				
				\label{performance_VS_age_marathon}
		
			\end{figure}								
		
		
			\begin{figure}[h]	
			
				\centering
				
				\includegraphics[scale=0.6]{../data_analysis/plots_for_paper/performance_VS_age_semi_marathon.pdf}
				
				\caption{Relation between runners' performance (time in minutes to complete the race) and age, for the most popular half-marathons (20 Km and 21 Km), color coded by gender.}
				
				\label{performance_VS_age_semi_marathon}
			
			\end{figure}								
			
		\subsubsection*{Temperature-performance relation}
		
			we don't have enough data (can be re-checked)...\\
			
			some reviews on the topic:\\
			\url{http://runningstrong.com/temperature.html}\\
			\url{http://believeperform.com/performance/the-effects-of-heat-on-sport-performance/}
			
	
	
		\subsection*{Geographical analysis}
	
			(to be included ??)(by Antonio \& Gr) \\			
			
			It is interesting to assess from which cities runners come, and how many of them, from the different locations. 
			In fig. \ref{origin_towns_dist} we report as an example the quite 
			broad\footnote{Fitted exponent: $ \alpha = 1.90 \pm 0.03 $} distribution of number of runners, coming from the 2005 municipalities reported for the 2016 Lausanne Marathon.
			
			\begin{figure}[h]	
				
				\centering
				
				\includegraphics[scale=0.6]{../data_analysis/plots_for_paper/origin_towns_dist.pdf}
				
				\caption{}
				
				\label{origin_towns_dist}
				
			\end{figure}								
	
		\subsection*{Network of runners}
	
				(to be included ??)
				(by Gr)
		
		\subsection*{Forecast of career advancement (??)}
	
				(not done yet)		\\
				\href{https://fivethirtyeight.com/features/tell-us-two-things-and-well-tell-you-how-fast-youd-run-a-marathon/}{nice article on fivethertyeight},
				pointing to one of the best/latest model \cite{vickers2016empirical}

	

\section*{Discussion}

%The Discussion should be succinct and must not contain subheadings.

\section*{Methods}

%Topical subheadings are allowed. Authors must ensure that their Methods section includes adequate experimental and characterization data necessary for others in the field to reproduce their work.

	\subsection*{Data parsing}
	
		@stefano 
		(remember to add the \textit{definition of runner})
	
        We parsed the runner data directly from the \url{datasport.com} website. 
        We collected more than 1.5 million entry from 2000 to 2015 in 223 distinct races.
        We did not focus on a particular race length. For example we included in our dataset 
        1km races for kids as well as ultra-marathon 200km races.

        Each entry refers to the performance of a runner in a race. It is composed of the following 
        fields:
        \begin{itemize}
            \item Race information. i.e. race name, race date, distance, race ID, temperature and weather.
            \item User information. i.e. name, sex, year of birth, user ID.
            \item Performance. i.e. time, pace, final rank and category.
        \end{itemize}
        Some of the fields were directly parsed from the webpage. Some others were derived. For
        example the sex is derived from the name of the person and from the category.
        In the following we show the techniques used.

        \paragraph{Parsing from \url{datasport.com}}
        We parsed for each race the ranking page sorted in alphabetical order. We found out that it
        is the page where the format is more consistent across the years.
        The webpage is visually formatted in a table-like structure as shown in Figure \ref{fig:exParsing}.
        However, the HTML code does not exploit a table structure. The content of an entry
        is all contained in a string. The string has to be parsed to extract all the fields.
        % for each entry and to develop an algorithm to automatically extract the different
        % fields from the string.
        The string format is not always consistent across the years.
        For example, in the early years some fields were missing.
        For this reason, we used a conservative approach in the parsing and in the post-processing phase 
        to reduce as much as possible the number of errors. 

        \paragraph{User ID} There is not a reliable way to extract the User ID from the webpage.
        We identify a \textit{runner} with the name and the year of birth. 
        This is the definition
        that in our tests reduces more the number of errors. 

        There are two different kind of errors in this approach:
        \begin{itemize} 
                \item The name can contain spell errors in some entry or it can be reported in the 
                    reversed order. We call this Type I error.
                    Our algorithm is not able to treat this kind of errors.
                    For example Abbot Mary born in 1974 and Mary Abbot born in
                    1974 are considered two different runners.
                    However, for our analysis this is not a serious problem.
                \item 2 different runners can have the same name and the same year of birth.
                    We call this Type II error.
                    This is a more serious error. 
                    For example, an algorithm that tries to predict the performance of a runner
                    will probably fail if 2 runners with different performance have
                    the same User ID.
        \end{itemize}

        We considered using also the Living Place in the User ID. However, this strongly increases
        the Type I error. For example, a runner can change his living place. Also, let us consider
        a runner that lives in Ecublens, near Lausanne. We found out that it is probable that he sometimes
        reports to live in Ecublens, sometimes in Ecublens VD, sometimes even in Lausanne.
        For this reason we think that the most robust User ID is the name and the year of birth.
        From a rough estimate we think that less than $0.5\%$ of the entries are affected by the Type II error.

        @ GIANROCCO devo spiegare come viene fuori 0.5\%? E' molto a caso come motivazione anche se credo la
        stima sia sensata.

        \paragraph{Sex} There is no information about the sex in the online dataset.
        The most reliable indicator for the sex of an entry is the category. For example, in 
        the category $M40$ there are only males with less than 40 years and this is true for
        each race where this category is present.
        An approach based only on the name of the person leads easily to errors since, for example,
        Andrea is both a male and female name.

        We determined the sex of each category. We used a name dataset that gives to
        each name the probability of being a male or female name.
        Averaging over all the names in the category we computed the male/female
        probability of the whole category. We only considered categories with a 
        male/female probability greater than $85\%$. This is due to the presence
        of mixed categories.
        We also considered categories with a significant number of entries. 1 or 2-entries categories
        present the same issues of an approach based only on the name of the person.

\begin{figure}[h!]
    \centering
    \includegraphics[scale=0.32]{Figures/exampleParsing.png}
    \caption{Example of parsed webpage from \url{datasport.com}.}
      % \kmyirmk{Just let latex do it's job. Don't bother moving the
      %   figures around.}}
    \label{fig:exParsing}
\end{figure}

		
	
	
	\subsection*{Data analysis}
	
		All analysis were performed on python notebooks (available on the \href{related repository}{related repository}), using standard python packages for data analysis and plotting, such as \texttt{pandas}, \texttt{seaborn}, \texttt{scipy}, \texttt{powerlaw}\footnote{\url{https://pypi.python.org/pypi/powerlaw}} and \texttt{networkx}.
	
	
	\subsection*{Data visualization}		
	
		We implemented interactive visualizations of some of our results and collected  them in the   \href{https://hopsuisse.github.io}{Hop Suisse}\footnote{\url{https://hopsuisse.github.io}} website.
		After exporting the data needed for the plot in \texttt{.json} dumps, we used \href{http://c3js.org}{C3.js} for the interactive plotting. More details on how datasets queries and plots were built can be found on the dedicated \href{https://github.com/hopsuisse/hopsuisse.github.io}{GitHub repository}\footnote{\url{https://github.com/hopsuisse/hopsuisse.github.io}}.
		We also build an \href{https://www.youtube.com/watch?v=MyvbnOXHShw}{animated infographics}\footnote{\url{https://www.youtube.com/watch?v=MyvbnOXHShw}}, inspired by  \href{https://en.wikipedia.org/wiki/Hans_Rosling}{Hans Rosling}'s work. With such video we wanted to show in a more powerful and clear way the relations among runners' mean pace, experience and age, providing as well information on gender and race length
		(the python code used to construct the video frames can be found in the \href{https://github.com/ggrrll/hop_suisse_ada_project_public/tree/master/8-video}{related folder}\footnote{\url{https://github.com/ggrrll/hop_suisse_ada_project_public/tree/master/8-video}} of our GitHub repository).
	

%\section*{Acknowledgements (not compulsory)}
%
%Acknowledgements should be brief, and should not include thanks to anonymous referees and editors, or effusive comments. Grant or contribution numbers may be acknowledged.

\section*{Author contributions statement}

G.L. and A.I. performed the data analysis.
S.S., O.C. and M.P. performed the data parsing.
G.L. and S.S. wrote the manuscript.
M.C. and M.S. review the manuscript.
%M.C. supervised the study.

\section*{Additional information}

All the code used to parse the data from \url{https://www.datasport.com/en/}, for data analysis and visualization can be found in our open GitHub repository: \url{https://github.com/ggrrll/hop_suisse_ada_project_public}.\\

\section*{Competing financial interests}

The authors declare no conflict of interests.


%%%%%% %%%%%% %%%%%% %%%%%%  BIBLIOGRAPHY %%%%%% %%%%%% %%%%%% %%%%%% %%%%%% 

\newpage
\bibliography{bib_hop_suisse}

%\noindent LaTeX formats citations and references automatically using the bibliography records in your .bib file, which you can edit via the project menu. Use the cite command for an inline citation, e.g.  \cite{Figueredo:2009dg}.


%To include, in this order: 
%\textbf{Accession codes} (where applicable); 
%\textbf{Competing financial interests} (mandatory statement). 
%
%The corresponding author is responsible for submitting a \href{http://www.nature.com/srep/policies/index.html#competing}{competing financial interests statement} on behalf of all authors of the paper. This statement must be included in the submitted article file.

%\begin{figure}[ht]
%\centering
%\includegraphics[width=\linewidth]{stream}
%\caption{Legend (350 words max). Example legend text.}
%\label{fig:stream}
%\end{figure}
%
%\begin{table}[ht]
%\centering
%\begin{tabular}{|l|l|l|}
%\hline
%Condition & n & p \\
%\hline
%A & 5 & 0.1 \\
%\hline
%B & 10 & 0.01 \\
%\hline
%\end{tabular}
%\caption{\label{tab:example}Legend (350 words max). Example legend text.}
%\end{table}

%Figures and tables can be referenced in LaTeX using the ref command, e.g. Figure \ref{fig:stream} and Table \ref{tab:example}.

\end{document}
